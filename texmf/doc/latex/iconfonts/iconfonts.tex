\DocumentMetadata{
  lang        = en,
  pdfversion  = 2.0,
  pdfstandard = ua-2,
  pdfstandard = a-4f,
  % testphase   = latest
}
\documentclass[10pt, a4paper]{ltxdoc}
\frenchspacing\setlength\parskip{0pt}\raggedbottom
\usepackage{\jobname}

\usepackage{geometry}
\usepackage{hologo}
\usepackage{luacode}
\usepackage{multicol}

\usepackage{fontspec}

\setlength{\columnsep}{.5cm}

\newcommand*{\iconlistlabel}[1]{\makebox[\labelwidth][c]{#1}}
\newenvironment{iconlist}
    {\begin{list}
        {}
        {%
            \setlength{\topsep}{0pt}%
            \setlength{\parskip}{0pt}%
            \setlength{\partopsep}{0pt}%
            \setlength{\leftmargin}{2.5em}%
            \setlength{\labelwidth}{1.25em}%
            \setlength{\labelsep}{1.25em}%
            \setlength{\itemindent}{0pt}%
            \setlength{\listparindent}{\parindent}%
            \setlength{\parsep}{0pt}%
            \setlength{\itemsep}{0pt}%
            \let\makelabel\iconlistlabel%
        }}
    {\end{list}}

\begin{document}
\GetFileInfo{\jobname.sty}

\title{The \textsf{\jobname} package\\Up-to-date \hologo{LuaLaTeX} support for \\\emph{Font Awesome Free} and \emph{Academicons}}
\author{Marc Penninga\thanks{URL: \href{https://github.com/drs-p/iconfonts}{\fontawesome{github} github.com/drs-p/iconfonts}}}
\date{Version \fileversion\ (\filedate);\\[3pt]
includes Font Awesome Free version \directlua{tex.sprint(iconfonts._fontawesome_version)}\ and Academicons version \directlua{tex.sprint(iconfonts._academicons_version)}}

\maketitle


\tableofcontents


\section{Introduction}
There already exist several packages on CTAN that provide access to the \emph{Font Awesome Free} and \emph{Academicons} icon fonts, but these have all become outdated and are missing many icons from newer versions of those fonts.
The \textsf{\jobname} package attempts to deal with the update problem by building the \hologo{LaTeX} support dynamically from data in the fonts themselves;
updating the package is then as simple as replacing the font files with newer ones.
The drawback of this approach is that the package only works with \hologo{LuaLaTeX}.


\section{Installation}
This package is not (yet?) available from CTAN, and hence also not included in \hologo{TeX}\,Live and other \hologo{TeX} distributions.
It must therefore be installed manually;
this is done by copying the contents of the \texttt{texmf/} directory to either your \texttt{\$TEXMFLOCAL} or your \texttt{\$TEXMFHOME} tree.
Installing it in \texttt{\$TEXMFLOCAL} will make the package available to all users on the system, but may require administrator rights;
installing it in \texttt{\$TEXMFHOME} makes it available only for your user.
The precise values of \texttt{\$TEXMFLOCAL} and \texttt{\$TEXMFHOME} depend on your \hologo{TeX} installation; see your installation's documentation for more information.

\pagebreak[2]
If you are using \hologo{TeX}\,Live (or another \hologo{TeX} distribution that uses the \textsf{kpathsea} library),
you can find the value of these variables by running
\begin{center}
    \verb|kpsewhich -expand-var '$TEXMFLOCAL'|
\end{center}
and similarly for \texttt{\$TEXMFHOME}.


\section{Usage}
The \textsf{\jobname} package contains all icons from the Academicons and Font Awesome Free fonts.
\DescribeMacro{\fontawesome}\DescribeMacro{\academicons}The package provides two commands to access those icons: \cs{fontawesome} and \cs{academicons}. Each of these has a single, mandatory argument: the \meta{name} of the icon.

Some Font Awesome icons have light or open variants, such as \fontawesome{face-smile} and \fontawesome{face-smile-o}.
Those variants are accessed by adding \texttt{-o} to the icon name: \verb|\fontawesome{face-smile-o}| instead of \verb|\fontawesome{face-smile}|.
A list of all icons and their names is included in the appendices.

The icons have varying widths.
\DescribeMacro{\fontawesome*}\DescribeMacro{\academicons*}Both commands have starred variants, \cs{fontawesome*} and \cs{academicons*}, that produce fixed-width icons (by centering them in a 1.25em wide box).


\section{Copyright}
\begin{itemize}
            \setlength{\topsep}{0pt}%
            \setlength{\parskip}{0pt}%
            \setlength{\partopsep}{0pt}%
            \setlength{\itemindent}{0pt}%
            \setlength{\listparindent}{\parindent}%
            \setlength{\parsep}{0pt}%
            \setlength{\itemsep}{0pt}%
    \item The Font Awesome fonts are Copyright 2025 Fonticons, Inc. (\url{https://fontawesome.com}), with Reserved Font Name “Font Awesome”.
    \item The Academicons font is Copyright 2023 James Walsh, Katja Bercic (\url{https://jpswalsh.github.io/academicons}), with Reserved Font Name “Academicons”.
    \item All other files are Copyright 2025 Marc Penninga.
\end{itemize}


\section{License}
Both the Font Awesome fonts and the Academicons font are released under the SIL Open Font License, version 1.1.
The rest of this package is released under the LaTeX Project Public License, version 1.3c.

All brand icons are trademarks of their respective owners.
The use of these trademarks does not indicate endorsement of the trademark holder by Font Awesome or the author, nor vice versa.
\textbf{Please do not use brand logos for any purpose except to represent the company, product, or service to which they refer.}


\newgeometry{hmargin={1cm,0cm}, vmargin={1cm,2cm}, footskip=1cm}
\appendix
\sffamily
\raggedright

\begin{luacode}
function print_icon(name, icon)
    tex.sprint(string.format("\\item[%s] %s", icon, name))
end
\end{luacode}

\section{List of Font Awesome icons}
\begin{multicols}{3}
    \begin{iconlist}
        \directlua{iconfonts._process_all_icons(print_icon, "solid", "regular")}
    \end{iconlist}
\end{multicols}

\clearpage
\section{List of Font Awesome brand icons}
\begin{multicols}{3}
    \begin{iconlist}
        \directlua{iconfonts._process_all_icons(print_icon, "brands")}
    \end{iconlist}
\end{multicols}

\clearpage
\section{List of Academicons icons}
\begin{multicols}{3}
    \begin{iconlist}
        \directlua{iconfonts._process_all_icons(print_icon, "academicons")}
    \end{iconlist}
\end{multicols}


\end{document}


%% end of file `iconfonts.tex'.

